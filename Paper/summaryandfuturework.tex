%!TEX root = ./main.tex
%
% This file is part of the i10 thesis template developed and used by the
% Media Computing Group at RWTH Aachen University.
% The current version of this template can be obtained at
% <http://www.media.informatik.rwth-aachen.de/karrer.html>.

\chapter{Summary and future work}
\label{summaryandfuturework}

%here

\section{Summary and contributions}
\label{summaryandfuturework.summary}

%here
We built and tested a prototype to investigate the applicability of four feedback modalities for the low-level underwater navigation context.
The prototype consists of diving goggles with an LED glued to it, a stretchable headband with a thermoelectric cooling element and vibration motor, a snorkel, and the electronics including an Arduino Uno.
Visual, tactile, thermal, auditory feedback was investigated for their perceptibility and comfort in underwater scenario.

Results have shown that thermal feedback is not well suited for underwater application since it was not recognizable by a large amount of participants in the first place.
It is heavily influenced by the water and perceptibility varies depending on the water temperature.
This, however, was not observed outside the water where only the technical limitations of the thermoelectric cooler prolongates the recognition by the user.
Additionally the huge amount of power used by the peltier cooling element makes it impractical to use in mobile environment.

Visual, vibration, and auditory feedback was perceived well and technically suited to provide underwater navigation cues.
Qualitative analysis via a questionnaire, answered by the participants, revealed that vibration feedback performs best on a subjective level.
It does not occupy any of the senses which used for diving in contrast to the LED.
Some users report that wearing the waterproof in-ear headphones was uncomfortable and that they are not used to wear those underwater.

\section{Future work}
\label{summaryandfuturework.futurework}
\index{future work|(}
Our study solely focused on how fast the respective feedback can be perceived underwater and how comfortable it is.
We did not yet investigate how accurate the feedback methods can communicate navigation cues in the field.
In the future we will drop the thermoelectric cooler due to its bad performance underwater and massive power consumption.
Furthermore we will tweak our prototype to incorporate a symmetrical amount of LEDs and vibration motors.
The exact amount has to be investigated with a separate user study.

Since the vibration feedback on the head might have an influence on the comfort in the long term, we suggest to implement the vibration motors directly in the diving equipment of scuba-divers.
Well accepted locations were investigated by \cite{Kiss:2018:NSM:3173574.3174191}.

Accuracy of the navigation cues is the most interesting measurement after proving the perceivableness.
To compare the performance of low level cues and more sophisticated approaches, like augmented reality diving goggles and precise auditory instruction via bone conduction headphones, further investigation in real world scenarios will be conducted.

To conduct studies in the field the prototype will be made wireless.
The primary challenges will be the waterproof incorporation of the power supply and electronics as well as establishing precise measurements.
The omission of real time observation and communication requires technology presented in chapter~\ref{relatedwork}.
Furthermore a reasonable way to provide 3D navigation cues has to be investigated with focus on comfort and accuracy similar to HapticHead by \cite{Kaul_HapticHead}.
%here


\index{future work|)}
