%!TEX root = ./main.tex
%
% This file is part of the i10 thesis template developed and used by the
% Media Computing Group at RWTH Aachen University.
% The current version of this template can be obtained at
% <http://www.media.informatik.rwth-aachen.de/karrer.html>.

\chapter{Evaluation}
\label{evaluation}
\index{evaluation|(}

In this chapter we will evaluate our prototypes with respect to the quantitative and qualitative aspects of the different feedback methods. We are interested to what extent the perception of feedback differs onshore versus underwater regarding time until the stimulus is perceived. Afterwards the user gives feedback in order to tell if the conditions make a difference with respect to preference. 


Since vibro-tactile and visual feedback is common today for several application ashore we let users test it onshore as well as under water. We drop sound in the underwater condition since it requires specific waterproof earphones or bone conduction headphones. Both exist and provide a comparable quality. The wireless connection however is not trivial underwater. Auditory feedback is broadly used in today's traffic and therefore not of particular interest in our case. 

\section{User Study}
The user study is divided into two groups. The first one is testing the prototype under normal condition while sitting on a chair where the other group is examining the feedback  underwater. 


 \index{evaluation|)}
