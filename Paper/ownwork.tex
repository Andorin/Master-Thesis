%!TEX root = ./main.tex
%
% This file is part of the i10 thesis template developed and used by the
% Media Computing Group at RWTH Aachen University.
% The current version of this template can be obtained at
% <http://www.media.informatik.rwth-aachen.de/karrer.html>.

\chapter{Hardware Prototype and Software }
\label{ownwork} 


In this chapter we present the construction of the hardware setup and the user study software. 
Furthermore we talk about the technical considerations regarding each component and their feasibility.


\section{System Design}
The aim of this thesis is to investigate the perception of several feedback modalities underwater and their feasibility for low level navigation cues.
We include visual, auditory, and tactile feedback in form of a LED, waterproof in ear headphones,  a vibration motor, and a thermoelectric cooling module.
The prototype has to incorporate these methods as unobtrusive and comfortable as possible in particular when they are  inactive. 
Electronic connections have to be waterproof, undisturbing, and failsafe.
Furthermore all components should be affordable to provide an advantage over commercial solutions.

To investigate the recognition times and comfort of each technique we built a prototype composed of one LED in the diving goggles as well as a waterproof vibration motor and a peltier cooling module in a stretchable headband.
The headphones are provided separately.

\paragraph{Visual Feedback}
To provide visual low level feedback we use a common red 5mm LED.
An issue regarding luminous light emitted by an LED is its proneness to water reflections.
These reflections change rapidly due to water undulation and exterior lighting.
The color of the surroundings influence it as well.
For example light blue tiles in a swimming pool render a blue LED almost undetectable.
To provide clear recognizable feedback we tested several colors underwater and came to the conclusion that red LED is better recognizable than other common LED colors.




\paragraph{Auditory Feedback}

\paragraph{Vibration Feedback}

\paragraph{Thermal Feedback}

\subsection{Hardware Setup}



\subsection{Testing}



\subsection{Safety}




\section{Software}
