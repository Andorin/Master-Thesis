%!TEX root = ./main.tex

% This file is part of the i10 thesis template developed and used by the
% Media Computing Group at RWTH Aachen University.
% The current version of this template can be obtained at
% <http://www.media.informatik.rwth-aachen.de/karrer.html>.

\loadgeometry{myAbstract}

\chapter*{Abstract\markboth{Abstract}{Abstract}}
\addcontentsline{toc}{chapter}{\protect\numberline{}Abstract}
\label{abstract}

%here (english version)
A lot of research is going into underwater global positioning systems since radio waves do not propagate underwater. However, the few underwater navigation approaches out there use bulky screen devices which are held in hand. This leads to constrained movement,  is rather distractive, and prone to varying brightness conditions. In this master thesis we describe the construction of a prototype, which incorporates several feedback methods, and its evaluation. We implement a vibration motor, a red LED, and a peltier element in diving goggles and a headband. Additionally waterproof in-ear headphones were used for auditory feedback. Since these devices are worn on the head they allow an unintrusive way to give low level directional cues. In a user study we evaluate the feedback methods ashore as a baseline and compare it to their performance underwater and gather additional qualitative feedback of the participants. 

\chapter*{\"Uberblick\markboth{\"Uberblick}{\"Uberblick}}
\addcontentsline{toc}{chapter}{\protect\numberline{}\"Uberblick}
\label{ueberblick}

%here (deutsche version)
Global Positioning Systeme für Navigation unter Wasser ist ein aktuelles Forschungsgebiet, da Radiowellen unter Wasser nicht übertragen werden. Die wenigen Systeme, die Unterwassernavigation zu einem gewissen Level umsetzen, nutzen Bildschirme, die in der Hand gehalten werden. Das ist behindert die Bewegungsfreiheit, ist eher ablenkend und die Sichtbarkeit ist beeinflusst durch sich ändernde Lichtverhältnisse. In dieser Masterarbeit wird die Konstruktion eines Prototypen beschrieben, der verschiedene Feedbackmethoden beinhaltet und deren Auswertung. Wir benutzen einen Vibrationsmotor, eine rote LED und  ein Peltierelement in einer Taucherbrille und Stirnband. Zusätzlich verwenden wir wasserfeste In-Ohr-Kopfhörer für Akustisches Feedback. Da diese Geräte direkt am Kopf getragen werden, ermöglichen sie einen unaufdringlichen Wege Richtungsangaben auf niedrigem Level zu vermitteln. In einen Benutzerstudie werten wir die verschiedenen Feedbackmethoden an Land aus und vergleichen die Leistung mit der unter Wasser. Zusätzlichen sammeln wir qualitatives Feedback der Teilnehmer.


\loadgeometry{myText}
