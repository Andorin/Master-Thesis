%!TEX root = ./main.tex
%
% This file is part of the i10 thesis template developed and used by the
% Media Computing Group at RWTH Aachen University.
% The current version of this template can be obtained at
% <http://www.media.informatik.rwth-aachen.de/karrer.html>.

\chapter{Introduction}
\label{introduction}

Navigation under water is a tricky and sometimes critical task when diving. 
So far, characteristics of the diving spot such as current and landmarks available for orientation determine if is suitable for beginners or whether more experience is necessary. 
For professional divers commercial navigation systems exist, which mostly consists of a (head-up) display giving precise information. 
These systems allow divers to keep track of their orientation even under difficult conditions, but the technology necessary is complex and expensive, as the omnipresent GPS localization does not work below the water surface. 
Systems as proposed by Taraldsen et al.~\cite{Taraldsen_UnderwaterGPS} use acoustic beacon systems, other systems use GPS for an initial calibration and rely on a inertial measurement unit (IMU) under water~\cite{Rossi_Performance}.
For less safety critical situations, e.g., guiding swimmers snorkeling through a scenic reef, less detailed feedback is necessary, and thus, the presentation of the navigation information can be different. 
In this paper, we investigate which modality is optimal to communicate such minimal cues for underwater navigation.
The technology added to the snorkel gear should also be minimal to achieve high acceptance and low distraction, fulfilling the vision of non-visible ubiquitous computing~\cite{Weiser:1993:CSI:159544.159617}.

We compared haptic, visual, and acoustic feedback which is well-researched for orientation cues above the surface. 
We attached vibration motors, Peltier elements for thermal feedback, and LEDs to the snorkel mask, and had the participants of our study wear waterproof earphones (cf.\ Fig.~\ref{fig:figure1}). 
We measured reaction times and asked for qualitative feedback.
While light and vibration achieve similar reaction times, vibration was perceived as being more comfortable. 
Thermal feedback performed poorly as it stayed unnoticed several times. 
